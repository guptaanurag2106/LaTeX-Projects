\documentclass{article}
\usepackage[utf8]{inputenc}
\usepackage{array}
\usepackage{wrapfig}
\usepackage{multirow}
\usepackage{tabularx}

\title{Tables}

\author{An example from Overleaf}

\begin{document}

\maketitle

\listoftables

\vspace{1cm}

%---------------------------------------------------------------------
%Simplest working example
\begin{center}
\begin{tabular}{ c | c |c }
\hline
  cell1 & cell2 & cell3 \\ 
  cell4 & cell5 & cell6 \\  
  cell7 & cell8 & cell9\\ 
  \hline
  
\end{tabular}
\end{center}


\vspace{2cm}

%---------------------------------------------------------------------
%Vertical lines as column separators
\begin{center}
\begin{tabular}{ | c | c | c | } 
  \hline
  cell1 & cell2 &  cell3 \\ 
  cell4 & cell5 & cell6 \\ 
  cell7 & cell8 & cell9 \\ 
  \hline
\end{tabular}
\end{center}
%---------------------------------------------------------------------

\vspace{2 cm}
%Horizontal lines as row separators
\begin{center}
 \begin{tabular}{||c |c |c |c||} 
 \hline
 Col1 & Col2 & Col2 & Col3 \\ [1.1 ex]
 \hline\hline
 1 & 6 & 87837 & 787 \\ 
 \hline
 2 & 7 & 78 & 5415 \\
 \hline
 3 & 545 & 778 & 7507 \\
 \hline
 4 & 545 & 18744 & 7560 \\
 \hline
 5 & 88 & 788 & 6344 \\ [1ex] 
 \hline
 \end{tabular}
\end{center}
%---------------------------------------------------------------------

\vspace{1cm}

%---------------------------------------------------------------------
%Example of fixed column width and text long (array package required)
\begin{center}
\begin{tabular}{ | m{7em} | m{1cm}| m{1cm} | } 
  \hline
 MMMMMMMM& cell2 & cell3 \\ 
  \hline
  cell1 dummy text dummy text dummy text & cell5 & cell6 \\ [7ex]
  \hline
  cell7 & cell8 & cell9 \\ 
  \hline
\end{tabular}
\end{center}
%---------------------------------------------------------------------

\newpage

%---------------------------------------------------------------------
%Example of table with fixed length (tabularx package required)
\begin{tabularx}{0.8\textwidth} { | >{\raggedright\arraybackslash}X | >{\centering\arraybackslash}X | >{\raggedleft\arraybackslash}X | }
   \hline
   item 11 & item 12 & item 13 \\[2ex]
   \hline
   item 21  & item 22  & item 23  \\
   \hline
\end{tabularx}


\newpage

%---------------------------------------------------------------------
%Example of the multicolumn command
\begin{table}[h!]
\caption{Multicolumn table}
\begin{tabular}{ |p{3cm}||p{3cm}|p{3cm}|p{3cm}|  }
 \hline
 \multicolumn{4}{|c|}{Country List} \\
 \hline
 Country Name	 or Area Name& ISO ALPHA 2 Code	&ISO ALPHA 3 Code&ISO numeric Code\\
 \hline
 Afghanistan	& AF	&AFG&	004\\
 Aland Islands&	AX	& ALA	&248\\
 Albania	&AL	& ALB&	008\\
 Algeria	&DZ	& DZA&	012\\
 American Samoa&	AS	& ASM&016\\
 Andorra&	AD	& AND	&020\\
 Angola&	AO	& AGO&024\\
  \hline
 \end{tabular}
 \end{table}
%---------------------------------------------------------------------

\vspace{1cm}

%---------------------------------------------------------------------
%Example of the multirow command (multirow package needed)}
\begin{table}[h!]
\caption{Multirow table}
\begin{center}
\begin{tabular}{ |c|c|c|c| } 
\hline
col1 & col2 & col3 \\
  \hline
  \multirow{3}{4em}{Multiple row} & cell2 & cell3 \\ 
  & cell5 & cell6 \\ 
  & cell8 & cell9 \\ 
  \hline
\end{tabular}
\end{center}
\end{table}
%---------------------------------------------------------------------

\newpage

%---------------------------------------------------------------------
%Referencing and captioning tables
The table \ref{table:1} is an example of a referenced \LaTeX{} element.

\begin{table}[h!]
\centering
 \begin{tabular}{||c c c c||} 
 \hline
 Col1 & Col2 & Col2 & Col3 \\ [0.5ex] 
 \hline\hline
 1 & 6 & 87837 & 787 \\ 
 2 & 7 & 78 & 5415 \\
 3 & 545 & 778 & 7507 \\
 4 & 545 & 18744 & 7560 \\
 5 & 88 & 788 & 6344 \\ [1ex] 
 \hline
 \end{tabular}
 \caption{Table to test captions and labels}
 \label{table:1}
\end{table}
%---------------------------------------------------------------------




\end{document}
